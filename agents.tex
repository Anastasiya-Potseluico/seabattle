\documentclass[a4paper,english,russian]{article}
\usepackage[left=3cm,right=1cm,
        top=2cm,bottom=2cm,bindingoffset=0cm]{geometry}
\usepackage{graphicx}
\usepackage{mathtools}
\usepackage{indentfirst}
\usepackage{multirow}
\usepackage[inline]{enumitem}
\usepackage[T1]{fontenc}
\usepackage[utf8]{inputenc}
\usepackage[russian]{babel}
\usepackage{listings}
\begin{document}
\begin{tabular}{|c|l|l|}
\hline    Лабораторная работа № 3 & Студент, Группа & Поцелуйко А.С., Клевцов В.А., ПрИн-1Н\\ \cline{2-3} 
                          Разработка агентоориентированной          & Преподаватель & Кульцова М.Б. \\ \cline{2-3} 
           системы принятия решений.                         & Оценка & \\ \hline 
\end{tabular}
\\
\section{Цель работы} % (fold)
\par Перед нами стояла задача, реализации агентоориентированой системы на основе JADE и JESS.
\section{Задание} % (fold)
\par Создание игры "морской бой" на основе агентов и рассуждения основанного на фактах.
\section{Реализация} % (fold)
\par Принципиальным решением был отказ он дублирования данныих и перевод все принимаех решений в Jess.
\par Суммарно было разработано 135 правил, позволяющих реализовать простой искуственный интеллект позволяющий играть в "морской бой".
\par Правило позволяют определить:
\begin{enumerate}
    \item 4х палубный корабль утонул(16)
    \item 3х палубный корабль утонул(16)
    \item 2х палубный корабль утонул(16)
    \item 1х палубный корабль утонул(64)
    \item закрасить клетки вокруг корабля помахами(что бы агенты не стреляли туда)(12)
    \item выбрать позицию следующего выстрела(9)
    \item определить результат выстрела(2)
\end{enumerate}
\par Вне правил было реализовано определение конца игры и связываение агентов.
\end{document}
